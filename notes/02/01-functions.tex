In programs we often need to do the same thing multiple times in different places. Because of this almost all programming languages have the concept of \textbf{functions}.
\\

A function is a block of code that performs a specific action. We usually give the function a name, which allows us to use it elsewhere in our code. A function can accept and return values. For example, a function called \texttt{authorize} might accept a username and password and return \texttt{true} if the combination is valid or \texttt{false} if not.


\section{Declaring Functions}

A function looks like this:\footnote{In fact, functions in JavaScript can \href{https://dmitripavlutin.com/6-ways-to-declare-javascript-functions/}{look a number of different ways}}:

\begin{minted}{javascript}
    // a function to add two numbers together
    // this function gets given two values
    function (a, b) {
        // it adds the two values together
        // and then returns the sum
        return a + b;
    }
\end{minted}

The above function isn't useful because we haven't given it a name (an \textbf{anonymous function}), so we don't have any way to use it in our code.

\pagebreak

But a function is a value in JavaScript\footnote{This is \textit{not} true in a lot of languages}, so, as with any other value, we can assign functions to variables:

\begin{minted}{javascript}
    let add = function (a, b) {
        return a + b;
    };
\end{minted}

We can now refer to the function using the \texttt{add} variable.


\section{Calling Functions}

Functions do not run until you \textbf{call} them. You call a function by using the function name followed by a pair of brackets.
\\

When you call it, you can send a function values (known as \textbf{arguments}) which it can use.
\\

For example, the \texttt{add} function takes two arguments:

\begin{minted}{javascript}
    add(1, 2); // call add, passing the values 1 and 2
    add(3, 7); // call add, passing the values 3 and 7
\end{minted}

The arguments are passed to the function in the order that they are given.\footnote{Also worth nothing that in JavaScript it's possible to pass too many or too few arguments and the code will still run - although it generally won't do what you want it to.} We can then use the passed values in the function:

\begin{minted}{javascript}
    // if we call add(1, 2) then inside the function
    // a = 1 and b = 2 as the values are passed in order
    let add = function (a, b) {
        return a + b;
    };
\end{minted}

In the above example \texttt{a} and \texttt{b} are effectively variables that are temporarily assigned the values that you pass to the function when you call it. As with any other variable you can call them whatever you like.

\pagebreak

Most the time functions will return a value:

\begin{minted}{javascript}
    let onePlusTwo = add(1, 2); // 3
    let threePlusFive = add(onePlusTwo, 5); // 8
\end{minted}

A function can contain as much code as you like, although well-written functions should only try to do one thing.
\\

A function that doesn't explicitly return anything returns \texttt{undefined}:

\begin{minted}{javascript}
    let oops = text => {
        text + "!";
    };

    let value = oops("Hello");
    console.log(value); // undefined
\end{minted}

If \texttt{undefined} sneaks into your calculations you're likely to start getting strange results.


\begin{infobox}{Arguments \& Parameters}
    The values that we pass to the function when we call it are the \textbf{arguments}. When we accept those values in the function declaration they are called \textbf{parameters}.

    \begin{minted}{javascript}
        // a and b are the function parameters
        // the variable names we use inside the function body
        let add = function (a, b) {
            return a + b;
        };

        // 1 and 2 are the arguments
        // the values we call the function with
        add(1, 2);
    \end{minted}

    This is a fairly technical distinction that you can easily get by without knowing. A lot of programmers will use the term ``arguments'' in both cases.
\end{infobox}


\section{Fat Arrow (\texttt{=>})}

In modern browsers we can use fat arrow (\texttt{=>}) to neaten up our function declarations:

\begin{minted}{javascript}
    let add = (a, b) => a + b;

    // works exactly as before
    let onePlusTwo = add(1, 2); // 3
    let threePlusFive = add(onePlusTwo, 5); // 8
\end{minted}

If the function does something simple then this can save quite a lot of boilerplate: you no longer need to write out \texttt{function} and as long as the function fits on one line you also don't need curly braces or a \texttt{return} - it automatically returns whatever the right-hand side evaluates to.
\\


If the function only takes one argument, you can even skip the brackets:

\begin{minted}{javascript}
    let square = n => n * n;

    // call add with an argument
    let twoSquared = square(2); // 4
    let fiveSquared = square(5); // 25
\end{minted}


You can still use \texttt{=>} for multi-line functions:

\begin{minted}{javascript}
    let sum41 = (a, b) => {
        let total = a + b;
        return total === 41;
    };
\end{minted}

We need to use curly braces in this instance - which also means we need to manually \texttt{return} a value.
\\

It's worth noting that fat arrow syntax is not identical to a traditional function declaration: it inherits its parent's \textbf{context}. However, this is a technicality and is unlikely to cause you any issues.
\\

Unless you have a good reason not to, you should use fat arrow syntax.

\pagebreak

\section{Examples}

\hypertarget{greet}{A function to greet someone}:

\begin{minted}{javascript}
    // greet takes one argument
    // multi-line, so we need curly-brackets
    let greet = name => {
        // get the current hour of the day
        let hour = new Date().getHours();

        if (hour < 12) {
            return "Good morning " + name;
        } else if (hour < 18) {
            return "Good afternoon " + name;
        }

        // when a function returns a value it stops running
        // so this will only ever run if the two return statements
        // above don't run
        return "Good evening " + name;
    };

    let greeting = greet("Ezra");
    console.log(greeting);
\end{minted}


A function to multiply three numbers:

\begin{minted}{javascript}
    // product takes three arguments
    let product3 = (a, b, c) => a * b * c;

    // call product, separating arguments with commas
    let result = product3(2, 3, 4); // 24
\end{minted}

\pagebreak

\begin{infobox}{Returning Booleans in Conditionals}
    If you ever have a conditional where you return \texttt{true} in one case and \texttt{false} in the other:

    \begin{minted}{javascript}
        if (expression) {
            return true;
        } else {
            return false;
        }
    \end{minted}

    \texttt{expression} should always be a boolean value, as it's a condition in the \texttt{if} statement. So you can return the condition (\texttt{expression}) instead of using a conditional:

    \begin{minted}{javascript}
        return expression;
    \end{minted}

    Likewise, you can always replace:

    \begin{minted}{javascript}
        if (expression) {
            return false;
        } else {
            return true;
        }
    \end{minted}

    with:

    \begin{minted}{javascript}
        return !expression;
    \end{minted}

\end{infobox}

\pagebreak

\section{Writing Functions}

Here's the process to go through when you're writing a function:

\begin{enumerate}
    \item Think of a sensible name for the function: a short way of describing its purpose
    \item Think about how many arguments the function needs to accept: this will depend on what you're trying to do
    \item Think about what type of thing the function should return
    \item Write out the boilerplate
    \item The thinky bit: Work about how to turn the arguments into the return value
    \item Test it with a few values you know the answer to
    \item Refactor (Optional): see if you can tidy up the code
\end{enumerate}


For example, a function to convert Fahrenheit to Celsius:

\begin{enumerate}
    \item Let's call it \texttt{celsius}
    \item It should take a single argument: a number (the temperature in Fahrenheit), let's call the parameter \texttt{fahrenheit}
    \item It should return a number
    \item First put in the boilerplate:
        \begin{minted}{javascript}
            let celsius = fahrenheit => {
                // needs to return a number: fahrenheit in celsius
            };
        \end{minted}
    \item To convert Fahrenheit into Celsius you need to take away 32 and divide by 1.8 (see \href{https://duckduckgo.com}{DuckDuckGo})
        \begin{minted}{javascript}
            let celsius = fahrenheit => {
                return (fahrenheit - 32) / 1.8;
            };
        \end{minted}
    \item Check it with a few values:
        \begin{minted}{javascript}
            celsius(45); // 7.222222
            celsius(32); // 0
        \end{minted}
    \item Refactor:
        \begin{minted}{javascript}
            let celsius = fahrenheit => (fahrenheit - 32) / 1.8;
        \end{minted}
\end{enumerate}

\quoteinline{Any fool can write code that a computer can understand. Good programmers write code that humans can understand}{Martin Fowler}


\section{Additional Resources}

\begin{itemize}[leftmargin=*]
    \item \href{https://eloquentjavascript.net/03_functions.html}{Eloquent JavaScript: Functions}
    \item \href{https://developer.mozilla.org/en-US/docs/Web/JavaScript/Reference/Functions}{MDN: Functions}
    \item \href{https://nick.scialli.me/first-class-functions-in-javascript/}{First-Class Functions in JavaScript}
    \item \href{https://jrsinclair.com/articles/2016/gentle-introduction-to-functional-javascript-intro/}{A Gentle Introduction to Functional JavaScript}
    \item \href{https://www.youtube.com/watch?v=8aGhZQkoFbQ}{What the heck is the event loop anyway?}
\end{itemize}
