\section{Scope}

The \textbf{scope} of a variable is the parts of code that you can use the variable name in.
\\

\texttt{let} and \texttt{const} are \textbf{block scoped}.\footnote{This is probably the most common form of scoping in programming languages} This means that the variable's value is only accessible from within the block in which it is declared (including any blocks within that block).
\\

Variables created without declaring them will go into \textbf{global} scope.

\begin{minted}{javascript}
    let x = 10;

    if (x === 10) { // new block
        let x = 20;
        y = 50; // goes into global scope, no variable declaration
        console.log(x); // 20 - different x
    }

    // logs 10: the x inside the if statement is scoped to that block
    console.log(x);
\end{minted}

Function parameters are always scoped to the function they belong to. Variables created with \texttt{var} are also scoped to the containing function, \textit{not} block scoped like \texttt{let} and \texttt{const}.

\begin{minted}{javascript}
    var x = 1; // in the global scope

    var fn = y => { // y is only available inside the function
        var z = 2; // z is only available inside the function

        if (z < 3) {
            var z = 4; // overwrites z above - var not block scoped
        }

        // can reference x, it was declared outside of the function
        return z + x + y;
    };

    var result = fn(12); // we can only access x, result, and fn here
\end{minted}


\section{Function Composition}
If you write a function that works, then don't be tempted to add additional functionality later. Instead, write a new function that calls the one you've already written.
\\

For example, say you had a function that works out if a number is divisible by 3. You then need to write a function that works out if a number is divisible by 3 \textit{and} a square number. Rather than editing your existing \texttt{divisibleBy3} function, create a new function that calls the \texttt{divisibleBy3} function.

\begin{minted}{javascript}
    // does number divide by 3 with no remainder
    let divisibleBy3 = n => n % 3 === 0;

    // is square root of number an integer
    let isSquare = n => Math.sqrt(n) % 1 === 0;

    // combine the two bits of functionality in one new function
    let divisibleBy3Square = n => divisibleBy3(n) && isSquare(n);
\end{minted}

If you get used to writing short functions that do one thing well, you'll find it much easier to perfect the art of \textbf{function composition}, arguably one of the most important skills in writing complex programs.


\section{Pure Functions}

You may sometimes hear people talk about \textbf{pure} functions. A pure function is one that fits the \textit{mathematical} concept of a function: given the same inputs you will always get back the same outputs.
\\

For example, a function like \texttt{add} is a \textit{pure} function because if you give call it with \texttt{add(4, 5)} you will \textit{always} get back \texttt{9}. There are no two inputs you can pass it that will sometimes return one answer but other times return something else.
\\

A pure function need not be doing maths to fit the mathematical concept. The function below is also pure, even though it deals with strings:

\begin{minted}{javascript}
    let shock = a => a + "!";
\end{minted}

The following function is pure too. It will return the string ``spooky'' for \textit{any} input:

\begin{minted}{javascript}
    let ghost = () => "spooky";
\end{minted}

The function \texttt{greet} (from the \hyperlink{greet}{previous chapter}) is \textit{not} pure: depending on the time of day it will return \textit{different} values given the same string. We \textit{could} make it pure by passing the current hour of the day in as a second argument, although this would make it harder to use.
\\

Pure functions also shouldn't cause \textbf{side-effects}: they should only work with the values they are given and not change anything else.
\\

The function below causes side-effects:

\begin{minted}{javascript}
    let x = 10;

    let thing = b => {
        x = b + 1;
        return x;
    };
\end{minted}

Even though this function will always return the same value given the same inputs, it changes the value of \texttt{x}, which lives outside the function.

\pagebreak

The following is also not pure:

\begin{minted}{javascript}
    let thing = b => {
        console.log(b);
        return b + 1;
    };
\end{minted}

Although this presumably doesn't change any values outside of the function it does call \texttt{console.log()}, which outputs text somewhere, thus changing \textit{the world itself} - which is a pretty major side-effect!
\\

It's worth noting that pure functions can call \textit{other} pure functions without losing their purity, so the \texttt{divisibleBy3Square} function in the previous section is pure, as both of the functions it calls are also pure.
\\

We can't write a useful program consisting \textit{entirely} of pure functions, as it wouldn't be able to output anything or respond to a user's input. But, because they are self-contained, pure functions are generally much easier to reason about. So it's a good idea to use pure functions wherever you can.

\pagebreak


\section{Recursive Functions}

\quoteinline{In order to understand recursion, one must first understand recursion}{Some smart-arse}

A function can call itself. This can be surprisingly powerful:

\begin{minted}{javascript}
    // works out the factorial of n
    let factorial = n => {
        // this condition stops the function calling itself forever
        if (n <= 1) {
            return 1;
        }

        // to find the factorial of n
        // times n by the factorial of n - 1
        // e.g. 5! = 5 * 4! = 5 * 4 * 3! etc.
        return n * factorial(n - 1);
    };
\end{minted}

Or if that's too long:

\begin{minted}{javascript}
    let factorial = n => (n <= 1) ? 1 : n * factorial(n - 1);
\end{minted}

A recursive function creates a looping structure.
\\

Anything we can do with recursion can also be done using traditional loops, but it's often much less elegant.
\\

As with other types of loop, we need to be careful not to create an infinite loop. If we didn't have the \texttt{n <= 1} check in the \texttt{factorial} function, it would keep going forever.

\quoteinline{A programmer walks into a bar and ask for a drink. The bartender says I'll give you a drink if you tell me a programmer joke. And he says: a programmer walks into a bar and ask for a drink. The bartender says I'll give you a drink if you tell me a programmer joke. And he says: a programmer walks into a bar and ask for a drink. So he gives the guy a drink, so he gives the guy a drink, so he gives the guy a drink.}{}


\section{Additional Resources}

\begin{itemize}[leftmargin=*]
    \item \href{https://scotch.io/tutorials/understanding-scope-in-javascript}{Scope in JavaScript}
    \item \href{https://dev.to/lydiahallie/javascript-visualized-scope-chain-13pd}{JavaScript Visualised: Scope}
    \item \href{https://www.youtube.com/watch?v=5LEuJNLfLN0}{Video: Scope in JavaScript}
    \item \href{https://2ality.com/2019/07/global-scope.html}{How do JavaScript’s global variables really work?}
    \item \href{https://medium.com/javascript-scene/master-the-javascript-interview-what-is-function-composition-20dfb109a1a0}{What is Function Composition?}
    \item \href{https://whatthefork.is/composition}{What the Fork Is: Composition}
    \item \href{https://medium.com/javascript-scene/master-the-javascript-interview-what-is-a-pure-function-d1c076bec976}{What is a Pure Function?}
    \item \href{https://www.sitepoint.com/recursion-functional-javascript/}{Recursion in Functional JavaScript}
    \item \href{http://jeffe.cs.illinois.edu/teaching/algorithms/book/01-recursion.pdf}{Algorithms: Recursion}: \textit{deep} dive into recursion
\end{itemize}
