\begin{itemize}[leftmargin=*]
    \item
        \textbf{Array}:
        a collection of items stored in a specific order, accessed via their position (index)
    \item
        \textbf{Boolean}:
        a value that is either true (\texttt{1}) or false (\texttt{0})
    \item
        \textbf{Class}:
        an abstract representation of a specific object structure which can be used to create object instances
    \item
        \textbf{Concatenation}:
        joining two strings together
    \item
        \textbf{Conditional}:
        a piece of code that will do different things depending on the value of a truth condition
    \item
        \textbf{Destructuring}:
        a way to store the value of specified object properties into variables
    \item
        \textbf{Expression}:
        a piece of code that is equal to some value - can be stored in a variable
    \item
        \textbf{Function}:
        a piece of code can be run zero or more times by calling it, often passing arguments and usually returning a value
    \item
        \textbf{Index}:
        the position of an item in an array, starting at \texttt{0} in most programming languages
    \item
        \textbf{Infinite Loop}:
        a loop with a looping condition that is always \texttt{true} so it never stops running
    \item
        \textbf{Interpolation}:
        inserting values from variables into a string
    \item
        \textbf{Iterator Methods}:
        array methods that run for each value in the array
    \item
        \textbf{Loop}:
        a piece of code that runs zero or more times depending on the value of a looping condition
    \item
        \textbf{Object}:
        a collection of items stored using named keys
    \item
        \textbf{Property}:
        a value stored on an object
    \item
        \textbf{Scope}:
        where a value is visible within code
    \item
        \textbf{String}:
        a sequence of characters
    \item
        \textbf{Turing Complete}:
        a programming language that can approximately simulate the computational aspects of any other programming language\footnote{\href{https://en.wikipedia.org/wiki/Turing\_completeness}{https://en.wikipedia.org/wiki/Turing\_completeness}}
    \item
        \textbf{Type}:
        the different sorts of things that a programming language understands (e.g. numbers, characters, lists)
    \item
        \textbf{Variable}: a
        way to store a value using a name
\end{itemize}
