\section{Strings}

Under the hood strings are actually objects. That means that strings have various properties and methods:

\begin{minted}{javascript}
    let str = "A String";

    // properties
    str.length; // 8

    // parts of a string
    str.charAt(2); // "S"
    str.substring(2, 5); // "Str" - start at index 2, end before index 5
    str.search("tr"); // 3 - found starting at index 3
    str.search("fr"); // -1 - didn't find it

    // change case
    str.toLowerCase(); // "a string"
    str.toUpperCase(); // "A STRING"

    // splitting
    str.split(" "); // split on space - ["A", "String"]
    str.split("ri"); // split on "ri" - ["A St", "ng"]
    str.split(""); // on each character - ["A"," ","S","t","r","i","n","g"]
\end{minted}

A full list of properties can be found on \href{https://developer.mozilla.org/en-US/docs/Web/JavaScript/Reference/Global_Objects/String}{MDN}.

\pagebreak

\section{\texttt{Date}}

The built-in \texttt{Date} object allows you to manipulate dates.

\begin{minted}{javascript}
    let now = new Date(); // a date object representing now

    // a date representing 5:08 am on 24th August 2018
    let birthdate = new Date("2018-08-24T05:08:00");

    birthdate.getFullYear(); // 2018
    birthdate.getDate(); // 24
    birthdate.getDay(); // 5 (0 - 6, 0 is Sunday and 6 is Saturday)
    birthdate.getMonth(); // 7 (0 - 11,  0 = January - c'est stupide!)
    birthdate.getTime(); // 1535083680000
\end{minted}

Generally it's easier to use a library like \href{http://momentjs.com}{moment.js}.


\begin{infobox}{The Beginning of Time}
    You might be wondering what the \texttt{1535083680000} value from \texttt{.getTime()} represents.
    \\

    It is, \textit{of course}, the number of milliseconds between the given date and 00:00 GMT on the 1st of January 1970.
    \\

    For an entertaining look at how computers handle dates, check out \href{https://zachholman.com/talk/utc-is-enough-for-everyone-right}{UTC is enough for everyone, right?}
\end{infobox}

\pagebreak

\section{\texttt{Math}}

The \texttt{Math} object lets you do more complex mathematical operations:

\begin{minted}{javascript}
    // Useful properties
    Math.PI; // 3.141592653589793
    Math.E; // 2.718281828459045

    // Rounding
    Math.floor(3.45); // 3
    Math.ceil(3.45); // 4
    Math.round(3.45); // 3

    // Exponents
    Math.sqrt(4); // 2
    Math.pow(2, 3); // 8

    // Other mathematical functions
    Math.log(6); // 1.791759469228055
    Math.cos(45); // 0.5253219888177297
\end{minted}

\pagebreak

\begin{infobox}{Putting the Java in JavaScript}
    The \texttt{Date} and \texttt{Math} objects both feel decidedly un-JavaScripty. That's because they were taken directly from Java.
    \\

    JavaScript was originally going to be called ``Mocha'' and then ``LiveScript''. But Netscape, the creators of JavaScript, settled on ``JavaScript'' after making a deal with Sun, the creators of Java. Sun agreed to give Netscape some money as long as they called their new language ``JavaScript'' to make it sound like a toy version of Java. Sun also insisted that JavaScript include the \texttt{Date} and \texttt{Math} objects from Java - despite the fact that Java and JavaScript have barely anything in common other than that they both ran in a browser.
    \\

    Ironically JavaScript went on to eclipse Java as the language of choice for apps that would run on any system.
    \\

    This also means that Oracle, who bought Sun in 2010, own the trademark on the name ``JavaScript''. That's why it's generally referred to as ECMAScript (after the European Computer Manufacturers Association, who control the standard) in any technical documentation.
    \\

    You can see the origins of the insane \texttt{getMonth()} method on the \href{https://docs.oracle.com/javase/7/docs/api/java/util/Date.html#getMonth()}{Java documentation}
\end{infobox}

\section{Additional Resources}

\begin{itemize}[leftmargin=*]
    \item \href{https://developer.mozilla.org/en-US/docs/Web/JavaScript/Reference/Global_Objects/Math}{MDN: \texttt{Math}}
    \item \href{https://developer.mozilla.org/en-US/docs/Web/JavaScript/Reference/Global_Objects/Date}{MDN: \texttt{Date}}
    \item \href{https://alligator.io/js/date-object/}{The JavaScript \texttt{Date} Object}
    \item \href{https://css-tricks.com/everything-you-need-to-know-about-date-in-javascript/}{Everything You Need to Know About Date in JavaScript}
    \item \href{https://medium.com/@toastui/handling-time-zone-in-javascript-547e67aa842d}{Handling Time Zone in JavaScript}
\end{itemize}
