Objects are \textit{unordered} collections of values. We use objects to collect together a set of related values. Objects have \textbf{keys} (names) and \textbf{properties} (values).
\\

Properties can be any type of value (numbers, strings, booleans, functions, arrays, and other objects). We tend not to iterate over the items in an object\footnote{Although we'll look at \textbf{maps} later, which are an exception to this}, so it's fine for different properties to store different types of values.

\begin{minted}{javascript}
    let empty = {}; // an empty object

    // an object representing a person
    let person = {
        name: "Kofi",
        age: 58,
        favouriteColours: ["purple", "green"],
        addresss: { // an object representing and address
            number: 54, // assign the value 54 to the property "number"
            road: "Park Street",
            postcode: "BS3 9LX",
        },
    };
\end{minted}

When we write objects out like this, we call them \textbf{object literals}. Later on we'll look at classes, which give us a different way of creating objects.

\pagebreak

\section{Reading Properties}

We use \textbf{dot-notation} to read the values of properties:

\begin{minted}{javascript}
    console.log(person.name); // "Kofi"
    console.log(person.favouriteColours); // ["purple", "green"]
\end{minted}

You can also use array-style notation (although dot-notation is preferred):

\begin{minted}{javascript}
    console.log(person["name"]); // "Kofi" - using array style notation
\end{minted}

\section{Writing Properties}

You can change the value of a property:

\begin{minted}{javascript}
    person.name = "Ban";
    person.age = 59;
\end{minted}

You can also use square bracket notation. This can be useful if you have the property name stored in a variable:

\begin{minted}{javascript}
    let property = "age";

    // elsewhere
    person[property] = 80;
\end{minted}

You can also add new properties to an object after it's been created:

\begin{minted}{javascript}
    // creates a new property, "country", and gives it "UK" as a value
    person.nationality = "Ghanaian";
\end{minted}

\begin{infobox}{Nested Properties}
    If we wanted to access the \texttt{road} property of the \texttt{address} property of the \texttt{person} object we can do this by extending the dot-notation:

    \begin{minted}{javascript}
        console.log(person.address.road); // "Park Street"
        person.address.road = "Bath Road";
        console.log(person.address.road); // "Bath Road"
    \end{minted}

    You can do this to any level in the hierarchy: just make sure you include \textit{every} level.
\end{infobox}


\section{Methods}

Objects can have functions as property values. We normally refer to these as \textbf{methods}. There is an object-specific way of writing functions:\footnote{You can write them using \texttt{=>}, but if you do you'll won't have access to \texttt{this}, which we'll cover shortly.}

\begin{minted}{javascript}
    let basicMaths = {
        pi: 3.141592654,

        add(a, b) {
            return a + b;
        },

        minus(a, b) {
            return a - b;
        },
    };

    basicMaths.add(1, 2); // 3
    basicMaths.minus(1, 2); // -1
    basicMaths.pi; // 3.141592654
\end{minted}

When you used \texttt{arr.sort()}, \texttt{str.charAt(4)}, etc. you were calling methods of arrays and strings.
\\

\textit{Methods are still properties}, so you can't have a method that has the same name as another property (e.g. you can't have a \texttt{.pi()} method and a \texttt{.pi} property)


\section{\texttt{this}}

Objects can refer to themselves using the \texttt{this} keyword:

\begin{minted}{javascript}
    let tori = {
        name: "Tori",
        birthdate: new Date("1984-11-22"),

        getAge() {
            let now = new Date();
            let millisecondsPerYear = 31556952000;

            // this.birthdate is the birthdate property above
            let years = (now - this.birthdate) / millisecondsPerYear;

            return Math.floor(years);
        }
    };

    tori.getAge(); // 34
\end{minted}

In the example above, it looks like we \textit{could} have used \texttt{tori} instead of \textit{this}. But objects aren't always assigned to named variables and what's stored in a specific variable can change, whereas the value of \texttt{this} will always point to the object it belongs to.\footnote{That's not strictly true, \texttt{this} is a bit broken in JavaScript. But we'll get to that in the React week}
\\

Because methods have access to an object's properties via \texttt{this} it is more common for them to not take any arguments. In the above example we don't need to pass anything to the \texttt{getAge()} method because all of the relevant data are stored as properties of the object.

\pagebreak

\section{Additional Resources}

\begin{itemize}[leftmargin=*]
    \item \href{https://eloquentjavascript.net/04_data.html}{Eloquent JavaScript: Objects and Arrays}
    \item \href{https://developer.mozilla.org/en-US/docs/Web/JavaScript/Reference/Global_Objects/Object}{MDN: Objects}
    \item \href{https://developer.mozilla.org/en-US/docs/Learn/JavaScript/Objects/Basics}{MDN: Object Basics}
    \item \href{https://blog.bitsrc.io/the-chronicles-of-javascript-objects-2d6b9205cd66}{The Chronicles of JavaScript Objects}
    \item \href{https://medium.com/javascript-scene/what-is-this-the-inner-workings-of-javascript-objects-d397bfa0708a}{What Is \texttt{this}? The Inner Workings of JavaScript Objects}
\end{itemize}
