We can also create abstract concepts (\textbf{classes}) of objects:

\begin{minted}{javascript}
    class Person {
        constructor(name, dob) {
            this.name = name;
            this.dob = dob;
        }

        getAge() {
            let now = new Date();
            let millisecondsPerYear = 31556952000;
            let years = (now - this.dob) / millisecondsPerYear;

            return Math.floor(years);
        }
    }
\end{minted}

This allows us to create multiple object \textbf{instances} with the same structure, without having to rewrite the same object literal each time.
\\

Each instance has its own set of properties and internally \texttt{this} refers to the specific instance it belongs to:

\begin{minted}{javascript}
    // we use the "new" keyword to create an "instance" of Person
    let kye = new Person("Kye", new Date("1965-03-19"));
    let jane = new Person("Jane", new Date("1973-11-27"));

    // we get back two separate ages, as they are different instances
    console.log(kye.getAge(), jane.getAge());
\end{minted}

Note that we give classes names with capital letters: \texttt{Person} rather than \texttt{person}. You don't have to, but it makes it much more obvious what your code is doing.


\section{An Example}

A book class:

\begin{minted}{javascript}
class Book {
    constructor(title, author) {
        this.title = title;
        this.author = author;
        this.price = null;
    }

    setPrice(value) {
        this.price = value;
        return this;
    }

    getPrice() {
        if (this.price === null) {
            return "Unknown";
        }

        return "£" + this.price.toFixed(2);
    }
}

let book = new Book("Lord of the Rrrings", "JRRRRR Tolkien");
console.log(book.getPrice()); // "Unknown"

book.setPrice(9.9);
console.log(book.getPrice()); // "£9.90"
\end{minted}

It's considered good practice to write \textbf{getter} and \textbf{setter} methods for reading and writing properties when using classes.
\\

In the example above if we just did \texttt{book.price} we'd get back \texttt{null} some of the time. Whereas if we write a \textbf{getter} function we can guarantee that it always comes back with a useful value that we can show to a user.


\begin{infobox}{Returning \texttt{this}}

    We often return \texttt{this} from \textbf{setter} methods: i.e. methods that accept values but don't have an obvious return value.

    \begin{minted}{javascript}
        class Book {
            // ...

            setPrice(value) {
                this.price = value;

                // return this from setPrice
                return this;
            }

            // ...
        }
    \end{minted}

    This allows us to \textbf{chain} together methods on an object:

    \begin{minted}{javascript}
        // because .setPrice returns the book object
        // we can then run .getPrice on it
        book.setPrice(99).getPrice();
    \end{minted}

    \textbf{Note}: it wouldn't make sense to return \texttt{this} from \textbf{getter} methods, as the whole point of a getter method is to return a specific value
\end{infobox}



\section{Additional Resources}

\begin{itemize}[leftmargin=*]
    \item \href{https://developer.mozilla.org/en-US/docs/Web/JavaScript/Reference/Classes}{MDN: Classes}
    \item \href{https://javascriptplayground.com/es5-getters-setters/}{ES5 Getters and Setters} - an alternative way to do getters/setters
    \item \href{https://developers.google.com/web/updates/2018/12/class-fields}{Public and private class fields}
\end{itemize}
