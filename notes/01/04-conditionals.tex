\section{\texttt{if} statements}

Computers need to be able to make decisions: if such-and-such is the case then do something. Without this ability we couldn't write programs that do anything other than basic calculations.
\\

We can use our logic rules in combination with \texttt{if} statements to decide what our program will do:


\begin{minted}{javascript}
    let value = 8;
    let max = 10;

    if (value <= max) {
        console.log("Valid"); // this will run
    }
\end{minted}

The part inside the \texttt{if} brackets is the \textbf{truth condition}\footnote{Hence ``conditional''} - this will always be evaluated as either \texttt{true} or \texttt{false}. If it evaluates to \texttt{true}, then the block of code will execute, if it evaluates to \texttt{false} then it won't.


\subsection{\texttt{else}}

You can also add an \texttt{else} block. If the conditional is \texttt{true} then the first block of code runs, otherwise the second block of code will run.

\begin{minted}{javascript}
    let username = "mark";

    // the truth condition
    if (username === "admin") {
        // this runs if it's true
        console.log("Hello Admin"); // won't run
    } else {
        // this runs if it's false
        console.log("Unauthorized!"); // will run
    }
\end{minted}


\subsection{\texttt{else if}}

You can also use \texttt{else if} blocks. These let you check multiple conditions. You can have as many of these as you like.
\\

As in the previous case, one and only one of these blocks of code will run. If the first condition is \texttt{false} then the first \texttt{else if} condition is checked; if that is also \texttt{false} then it will move onto the next; and so on until it reaches the final \texttt{else} block, which will only run if \textit{all} of the previous conditions have returned \texttt{false}.

\begin{minted}{javascript}
    let average = (10 + 13 + 15) / 3;

    if (average <= 10) {
        console.log("Less than 10"); // won't run
    } else if (average < 20) {
        console.log("Less than 20"); // will run
    } else {
        console.log("20 or more");   // won't run
    }
\end{minted}

\pagebreak

If we hadn't included the final \texttt{else} block, then it's possible that none of the code blocks would run:

\begin{minted}{javascript}
    let average = (30 + 50 + 100) / 3;

    if (average <= 10) {
        console.log("Less than 10"); // won't run
    } else if (average < 20) {
        console.log("Less than 20"); // won't run
    }
\end{minted}


\begin{infobox}{Truthy \& Falsey Values}
    Because JavaScript isn't that fussy about types, if you try and use a non-boolean value as a condition it will do its best to make things work:

    \begin{minted}{javascript}
    let value = "Hello";

    if (value) {
        // even though the condition isn't a boolean value
        // this will run as a non-empty string is "truthy"
    }
    \end{minted}

    Falsey values (ones that type cast to \texttt{false}) in JavaScript are:

    \begin{itemize}
        \item \texttt{false}
        \item \texttt{0}
        \item \texttt{""} (the empty string)
        \item \texttt{null}
        \item \texttt{undefined}
        \item \texttt{NaN}
    \end{itemize}

    Everything else is truthy (i.e. it type casts to \texttt{true}).
\end{infobox}

\pagebreak

\section{The Ternary Operator}

If the blocks of your conditional are both short, it can sometimes save space to use the \textbf{ternary operator}.
\\

Unlike an \texttt{if} \textbf{statement} the ternary operator is an \textbf{expression}, meaning that it equates to a value.

\begin{minted}{javascript}
    let current = 3;

    // set the value of next, dependent on the value of current
    let next = current > 5 ? 0 : current + 1; // next is set to 4
\end{minted}

Does the same as:

\begin{minted}{javascript}
    let current = 3;
    let next;

    if (current > 5) {
        next = 0;
    } else {
        next = current + 1;
    }
\end{minted}

The ternary operator consists of three parts:

\begin{enumerate}
    \item The condition (before the \texttt{?})
    \item The \textit{if true} result (before the \texttt{:})
    \item The \textit{if false} result (after the \texttt{:})
\end{enumerate}

In other words, if the bit before the \texttt{?} is true, then evaluate to the bit before the \texttt{:}, otherwise evaluate to the bit after the \texttt{:}.

\pagebreak

\begin{infobox}{Statements \& Expressions}
    An \textbf{expression} is any bit of code that equates to some value. If you could assign the whole bit of code to a variable, then it's an expression. An expression can be made up of sub-expressions.
    \\

    For example, all of the following equate to a value, so they are expressions:

    \begin{minted}{javascript}
        // equates to 46
        12 + 34;

        // equates to true
        ((true && false) || false) === false;

        // equates to some number, assuming celsius is a number
        (celsius * 9/5) + 32;
    \end{minted}

    A \textbf{statement} is just an instruction. Things like \texttt{if} statements or variable assignments: they do something, but aren't equal to a value. Statements can be made up of other statements and expressions. All expressions are also statements.

\end{infobox}

\pagebreak

\section{\texttt{switch} Statements}

If your \texttt{if} statement is just a series of checks against the same value, then a \texttt{switch} statement can sometimes save space:


\begin{minted}{javascript}
    let username = "admin";

    switch (username) {
        case "admin": console.log("Authorized"); break;
        case "Admin": console.log("Authorized"); break;
        case "Jeff": console.log("Back door! Authorized"); break;
        default: console.log("Unauthorized");
    }
\end{minted}

We use the \texttt{switch} keyword and then in brackets we put the value we want to check against. Inside the block, we list a series of \texttt{case} statements: if the value in the \texttt{switch} matches a specific \texttt{case} then that case statement will run. It will only run the first matching \texttt{case}. We can also have a \texttt{default} case which works much like an \texttt{else} block: it will run if none of the case statements are true.
\\

\texttt{switch} statements are neater than multiple \texttt{else if} statements, but make sure you remember to use \texttt{break} at the end of each \texttt{case}, otherwise you'll get unexpected behaviour\footnote{If you don't use \texttt{break} then every \texttt{case} statement below the one that was true will also run. This is called \textbf{fall-through} and is occasionally useful, but if you see it in code it's hard to know whether it's deliberate or not, so it's best not to use it}.

\pagebreak

\begin{infobox}{Indenting}

    You'll notice that all of the code inside the \texttt{if} blocks is indented:

    \begin{minted}{javascript}
        // anything between the opening and closing
        // curly braces should be indented
        if (x === 10) {
            console.log("x is 10"); // indented
        } // <-- first block block stops here, so don't indent anymore

        // new block, so indent everything inside the braces
        if (x < 20) {
            console.log("Less than 20"); // indented

            // a new block, so indent another level
            if (average < 20) {
                console.log("Less than 20"); // indented two times
            } // <-- end of a block, indent one less
        } // <-- end of outermost block, so stop indenting

        console.log("Less than 20"); // <-- outside of blocks, not indented
    \end{minted}

    This makes it clear which bits of the code are part of the if blocks and which bits are not.
    \\

    Basically, when you get to a \texttt{\{} you should indent another level and when you get to a \texttt{\}} you should de-indent.\footnotemark Your text editor will probably do this for you and, at this stage, it probably knows best.
    \\

    \textit{Fix indenting problems as soon as you spot them} - it will save you time later!
\end{infobox}

\footnotetext{Unindent? Outdent? Dent?}


\section{Additional Resources}

\begin{itemize}[leftmargin=*]
    \item \href{https://developer.mozilla.org/en-US/docs/Learn/JavaScript/Building_blocks/conditionals}{MDN: Conditionals}
    \item \href{https://eloquentjavascript.net/02_program_structure.html}{Eloquent JavaScript: Program Structure}
    \item \href{http://exploringjs.com/impatient-js/ch_control-flow.html}{JavaScript for Impatient Programmers: Control Flow}
\end{itemize}
