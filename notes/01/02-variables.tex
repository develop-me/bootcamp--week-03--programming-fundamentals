Variables are a way of storing a value using a name so that we can refer to it later.
\\

This serves two purposes:

\begin{enumerate}
    \item We can store the results of complex calculations so that we only have to calculate them once
    \item If we're sensible about what we call our variables it makes our code much easier to follow
\end{enumerate}

Before we use a variable we must \textbf{declare} it using the \texttt{let} keyword:

\begin{minted}{javascript}
    let email;
    let age;
\end{minted}

Declaring a variable lets JavaScript know that from that point on, if we use that series of characters in our code, it represents a value.
\\

Once we've declared the variable, we can \textbf{assign} it a value:

\begin{minted}{javascript}
    email = "orb@is.horse";
    age = 32;
\end{minted}

\pagebreak

We only need to declare a variable once. From that point on we can reassign the value if we want to:

\begin{minted}{javascript}
    // elsewhere
    email = "farm@wisdom.com";

    // elserwhere
    email = "shrimp.heaven@now.plumbing";
\end{minted}

Once we've stored a value in a variable we can use it to represent that value elsewhere in our code:

\begin{minted}{javascript}
    let pointless;
    pointless = email + age; // "shrimp.heaven@now.plumbing32"
\end{minted}

Generally we declare and assign variables at the same time:\footnote{If you see old JavaScript code that uses \texttt{var}, you will often find all the variables declared at the top of the file and values assigned to them later. This is because of something called ``\href{https://scotch.io/tutorials/understanding-hoisting-in-javascript}{hoisting}''. Luckily it's not necessary if you use \texttt{let}.}

\begin{minted}{javascript}
    let name = "Archie";

    let age = 4, // can also declare multiple variables in one go
        houseNumber = 21;

    // using variables
    let notUseful = age + houseNumber; // 25
\end{minted}


\pagebreak

\begin{infobox}{Variable Types}
    \texttt{let}\\
    The most commonly used in modern JS. Use this unless you can think of a good reason not to.

    \begin{minted}{javascript}
        let value = 10;
    \end{minted}

    \texttt{const}\\
    Useful if you want to make sure a value can't be changed, for example if you had a variable that stored some configuration.

    \begin{minted}{javascript}
        const maxVolume = 10;
        maxVolume = 11; // Error - can't assign a new value to maxVolume
    \end{minted}

    \texttt{var}\\
    Very common in older JS. Works almost identically to \texttt{let} except when it comes to \textbf{scoping}. Stick to \texttt{let} unless you're dealing with legacy code.

    \begin{minted}{javascript}
    var meh = 10;
    \end{minted}

\end{infobox}



\section{Naming Variables}

We can call a variable pretty much anything we want, but it's best to pick something that represents its purpose.

\begin{minted}{javascript}
    let name = "Ben"; // good
    let a = 394; // possibly ok for short bits of code
    let aRidiculouslyLongVariableName = 83; // maybe a bit long
    let appleSauce = 394; // huh?
    let name = "Not Ben"; // already used that...
\end{minted}

A variable name can contain:

\begin{itemize}
    \item alphanumeric characters
    \item underscores
    \item the dollar sign
\end{itemize}

It cannot:

\begin{itemize}
    \item contain spaces
    \item contain hyphens
    \item start with a number
    \item be a \href{https://developer.mozilla.org/en-US/docs/Web/JavaScript/Reference/Lexical_grammar#Keywords}{reserved word} (e.g. \texttt{class}, \texttt{let}, \texttt{var}, \texttt{function}, \texttt{if}, and many more)
\end{itemize}

We tend to use camel-case (lowercase first letter, uppercase beginnings of words - \texttt{likeThis}) - as opposed to snake-case (all lowercase, underscores between words - \texttt{like\_this}). You don't have to, but if you don't your code will look weird to everyone else and your friendship group will slowly dwindle.
\\

If we pick our variable names sensibly then it's easy to see what our code does:

\begin{minted}{javascript}
    let username = "potus";
    let password = "00000000";

    armNuclearWeapons(username, password);
\end{minted}

If we pick our variables names poorly it can be impossible to work out what's going on:

\begin{minted}{javascript}
    let a = "potus";
    let b = "00000000";

    doWhatevs(a, b); // we just started a thermonuclear war - oops!
\end{minted}

\pagebreak

\begin{infobox}{Comments}
    It's a good idea to explain unusual parts of your code. You can do this with comments.
    \\

    If you put \texttt{//} on a line in JavaScript then everything after that will be ignored when your code runs:

    \begin{minted}{javascript}
        // The number of milliseconds in a year
        // 1000 * 60 * 60 *  24 * 365.2425
        let millisecondsPerYear = 31556952000;
        let another = 12345; // you can comment at the end of a line too
    \end{minted}

    You can also do multi-line comments using \texttt{/*} and \texttt{*/}. Everything between the opening \texttt{/*} and the closing \texttt{*/} will be ignored.
    \\

    This can be useful if you need to temporarily disable a bit of code. But \textit{make sure you don't leave unused code lying around once everything is working}.

    \begin{minted}{javascript}
        /*
         * The number of milliseconds in a year
         * Calculated using: 1000 * 60 * 60 24 * 365.2425
         * Required for date calculations
         */
        let millisecondsPerYear = 31556952000;
    \end{minted}

    The extra \texttt{*} at the beginning of each line isn't necessary - but it looks nicer.
    \\

    If you change a bit of code, \textit{make sure you update the corresponding comments}: old/incorrect comments are worse than no comments at all.
\end{infobox}

\pagebreak

\section{Template Strings}

We often want to include something stored in a variable as part of a string.
\\

One option would be to concatenate the variables:

\begin{minted}{javascript}
let name = "Chetna";
let greeting = "Hello " + name + ", how are you?";
\end{minted}

However, rather than using quotation marks, we can put backticks (\texttt{`}) around our strings. This allows us to \textbf{interpolate} values:

\begin{minted}{javascript}
let name = "Chetna";
let greeting = `Hello ${name}, how are you?`;
\end{minted}

As you can see, we use \texttt{\$\{variable\}} inside the backticks to insert the value contained in a variable.

\section{Additional Resources}

\begin{itemize}[leftmargin=*]
    \item \href{https://eloquentjavascript.net/01_values.html}{Eloquent JavaScript: Values, Types, and Operators}
    \item \href{http://exploringjs.com/impatient-js/ch_variables-assignment.html}{JavaScript for Impatient Programmers: Variables \& Values}
    \item \href{https://developers.google.com/web/updates/2015/01/ES6-Template-Strings#string_substitution}{Template Strings}
    \item \href{https://github.com/kettanaito/naming-cheatsheet}{Naming Cheatsheet}
\end{itemize}
