We use a loop when we want to do something similar more than once.

\section{\texttt{for} Loops}

\texttt{for} loops are useful when you know how many times the loop should run.
\\

They consist of three parts:

\begin{enumerate}
    \item \texttt{let i = 0}: setup a variable that we use as a counter
    \item \texttt{i < 10}: keep running the loop as long as this is true
    \item \texttt{i += 1}: increment \texttt{i} by 1 every time the loop runs
\end{enumerate}

\begin{minted}{javascript}
    // will keep running until i is 9
    for (let i = 0; i < 10; i += 1) {
        console.log(i);
    }

    // 0, 1, 2, 3, 4, 5, 6, 7, 8, 9
\end{minted}

It is traditional to use \texttt{i} as the counter variable - one of the few places it's good practice to use a single letter variable name. If you need a loop inside a loop then just keep going down the alphabet (\texttt{j}, \texttt{k}, \ldots).


\pagebreak

\section{\texttt{while} Loops}

We use a \texttt{while} loop if we aren't sure how many times the loop needs to run.
\\

For example, say we wanted to add 1 to 2 to 3 to 4 and so on until we get to a number bigger than 100. We known when we want it to stop (when the total is bigger than 100), but we don't know how many times it needs to run.

\begin{minted}{javascript}
    let i = 0;     // keeps track of which number we're adding
    let total = 0; // keeps track of the total so far

    // will keep running until total is more than 100
    while (total <= 100) {
        total += i;
        i += 1;
    }

    console.log(total); // 105
    console.log(i); // 15 - so the loop ran 15 times
\end{minted}



\section{Infinite Loops}

We need to be careful to avoid infinite loops: loops that never stop running. These sometimes occur because of typos, but more often because you use a variable that isn't equal to what you're expecting.

\begin{minted}{javascript}
    for (let i = 0; i < 10; i -= 1) {
        // will never stop
        // why not?
    }
\end{minted}

An infinite loop will keep running until you kill the process that's running it. If you're running code in \texttt{node} and you think you've got an infinite loop then press \texttt{Ctrl+C}, which will kill the \texttt{node} process.

\pagebreak

\section{Additional Resources}

\begin{itemize}[leftmargin=*]
    \item \href{https://developer.mozilla.org/en-US/docs/Web/JavaScript/Guide/Loops_and_iteration}{MDN: Loops}
    \item \href{https://eloquentjavascript.net/02_program_structure.html}{Eloquent JavaScript: Program Structure}
    \item \href{http://exploringjs.com/impatient-js/ch_control-flow.html}{JavaScript for Impatient Programmers: Control Flow}
    \item \href{https://www.wnycstudios.org/story/radiolab-loops}{Radiolab: Loops}: Nothing to do with programming loops, but really interesting!
\end{itemize}
